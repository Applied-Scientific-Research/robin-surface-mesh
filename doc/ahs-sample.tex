\documentclass{ahs}

\usepackage{cite}

\jname{Journal of the American Helicopter Society}
\logno{xxxx}

\begin{document}

\markboth{Author Name}{}%

\title{Preparation of Papers for the \\ Journal of the American Helicopter~Society}

\author%[author.eps]
{First A. Author\thanks{Footnote text should indicate if paper was presented at National Meeting or other conference.}}
\affiliation{official title,\\ company affiliation,\\ and simple address (city and state)}
\author{Second B. Author, Jr.}
\affiliation{official title,\\ company affiliation,\\ and simple address (city and state)}

\maketitle

\begin{abstract}%
This is an example of abstract text. A helicopter can be defined as any flying machine using rotors to 
provide lift, propulsion and control forces. The rotor produces a lift 
force equal to the weight of the helicopter and because the generation 
of this lift force does not require any forward flight speed, the 
helicopter can rise vertically from the ground and hover. A simpler 
definition, therefore, is that a helicopter is an aircraft using a 
rotor (or rotors) that can hover.
\end{abstract}

\begin{nomenclature}[$C_{p}$]
\nomenentry{$A$}{amplitude of oscillation}
\nomenentry{$a$}{cylinder diameter}
\nomenentry{$C_{p}$}{pressure coefficient}
\nomenentry{$C_{x}$}{force coefficient in the $x$ direction}
\nomenentry{$C_{y}$}{force coefficient in the $y$ direction}
\nomenentry{c}{chord}
\nomenentry{d$t$}{time step}
\nomenentry{{\it Fx}}{$X$ component of the resultant pressure force acting on the vehicle}
\nomenentry{{\it Fy}}{$Y$ component of the resultant pressure force acting on the vehicle}
\nomenentry{$f, g$}{generic functions}
\nomenentry{$h$}{height}
\nomensection{Subscript}
\nomenentry{$i$}{time index during navigation}
\nomenentry{$j$}{waypoint index}
\nomenentry{$K$}{trailing-edge (TE) nondimensional angular deflection rate}
\end{nomenclature}

\section{Introduction}

This document is a \LaTeXe\ template. If you are reading a hardcopy
or PDF version of this document, please download the \LaTeXe\ macros
and sample file, \verb|ahs.cls|, from
\verb|http://www.vtol.org/index.html| so you can use it
to prepare your manuscript. The \LaTeXe\ class file uses only
Computer Modern (CM) typefaces, which are part of the \TeX/\LaTeX\
installation. Use the \LaTeX\ files for formatting purposes, but
please use \verb|ahs.cls| or \verb|ahs.pdf| for specific
layout instructions.  Authors will first need to save the
\verb|ahs.cls| to a current/working directory or TeX default
directory. Use italics for emphasis; do not\nobreak underline.

For references, which will first appear in the Introduction, AHS uses the style ``...as shown in
Johnson (Ref.~\citen{Johnson80a})..." or the other style is ``...which is given in
Ref.~\citen{Leishman00}." When we deal with a range of citations it is best to use the style  ``...as
shown in several sources including Refs.~\citen{Johnson80a,Leishman00,Friedmann93,Chopra85}." The use
of the ``cite.sty" package is encouraged, which can be included in the premable to your  \LaTeX\
document.

\section{Detailed Formatting Instructions}

The styles and formats for the AHS Papers class file have been
incorporated into the structure of this document using:-
\begin{verbatim}
   \documentclass{ahs}
\end{verbatim}
If you are reading a hardcopy or PDF version of this document, please
download the \LaTeXe\ macros and sample file, \verb|ahs.cls|,
from \verb|http://www.vtol.org/index.html| so you
can use it to prepare your manuscript. Use the \LaTeXe files for formatting
purposes, but please use \verb|ahs.cls| or \verb|ahs.pdf| for
specific layout instructions.  Authors will first need to save the file
\verb|ahs.cls| to a current/working directory or TeX default
directory.

\subsection{Document text}

The default font for AHS papers is only Computer Modern (CM)
typefaces (12-point size), which are part of the \TeX/\LaTeX\
installation. 

\subsection{Coding your document} 

Most commands described in this guide are not part of the standard
\LaTeX\ package.  The syntax/usage of some commands has been changed,
and some commands are newly defined to accommodate the AHS paper
style.  Such changes are explicitly mentioned in the sections where
these commands are described. 


\subsection{Title, authors' names, and affiliations}

The title of your paper should be coded as \verb|\title{Title..}|,
with capital and lower-case letters, and centered at the top of the
page.  The names of the authors, business or academic affiliation,
city, and state/province should follow on separate lines below the
title. The names of authors with the same affiliation can be listed
on the same line above their collective affiliation information.
The author name must be coded as \verb|\author{}|.
Author names are centered, and affiliations are centered and in
italic type. The affiliation line for each author is to include that
author's city, state, and zip/postal code (or city, province,
zip/postal code and country, as appropriate). The corresponding
affiliation is coded as
\verb|\affiliation{Author's Affiliation}|, which
should immediately follow the \verb|\authorname{}| command.
The title, author name, and affiliation
must be followed by the command \verb|\maketitle|.
Thus,

\begin{verbatim}
   \documentclass{ahs}
   \begin{document}
   \markboth{Author Name}{Article Title}
   \title{Article Title}
   \author{Name of Author 1}
   \affiliation{Affiliation of Author 1}
   \author{Name of Author 2}
   \affiliation{Affiliation of Author 2}
   \maketitle
\end{verbatim}
\verb|\affiliation{}| command is not part of the standard \LaTeX\ package.

The first footnote (lower left-hand side) is to contain the job title
and department name and street address/mail stop, for each author. 
This can be coded inside \verb|\author{}| command as
below:- 
\begin{verbatim}
   \author{Author Name\thanks{Insert Job Title, 
      Department Name, Address 
      for first author.}}
   \affiliation{Business or Academic Affiliation 1, 
      City, State, Zip Code}
\end{verbatim}

\subsection{Sections}

Use \verb|\section{}| for main section and \verb|\subsection{}| for subsections.

\subsection{Headings}

AHS style provides 3 levels of section headings and they are all
defined in the \verb|ahs.cls| class file:
\begin{itemize}
\item Heading 1 -- \verb"\section": bold 12-point font, centered.

\item Heading 2 -- \verb"\subsection": bold, flush left.

\item Heading 3 -- \verb"\subsubsection": bolditalic, flush left \& first line of the paragraph.

%\item Heading 4 -- \verb"\paragraph": italic, run-on with text. 
\end{itemize}

\subsection{Abstract}

The abstract should appear at the beginning of your paper. It should
be one paragraph long (not an introduction) and complete in itself
(no reference numbers). It should indicate subjects dealt with in the
paper and state the objectives of the investigation. Newly observed
facts and conclusions of the experiment or argument discussed in the
paper must be stated in summary form; readers should not have to read
the paper to understand the abstract. The abstract should coded after
\verb|\maketitle| command as below:
\begin{verbatim}
   \maketitle
   \begin{abstract}
   This is an example of abstract text.  This is an
   example of abstract text.  This is an example of
   abstract text.
   \end{abstract}
\end{verbatim}


\subsection{Images, figures, and tables}

All artwork, captions, figures, graphs, and tables will be reproduced
exactly as submitted. Be sure to submit any figures, tables,
graphs, or pictures as you want them printed. Company
logos and identification numbers are not permitted on your
illustrations.

Place figure captions below all figures; place table titles above the
tables. If your figure has multiple parts, include the labels ``a),''
``b),'' etc. below and to the left of each part, above the figure
caption. Please verify that the figures and tables you mention in the
text actually exist. {\it Please do not include captions as part of the figures}.

When citing a figure in the text, use the abbreviation ``Fig.~\citen{Johnson80a}'' except
at the beginning of a sentence where you should use ``Figure~\citen{Johnson80a}" . Do not abbreviate
``Table.'' Number each different type of illustration (i.e., figures, tables, images)
sequentially with relation to other illustrations of the same type.

Figures and tables are referred to as `Floats' in \LaTeX, reflecting
their floating nature. They are typically numbered whether or not
they have a caption and they are floated to the first available
position near the first reference to that figure/table within the
text.  This is accomplished by placing the figure/table command
immediately after the paragraph in which they are referred to for the
very first time.

An example of coding \verb|{figure}| is given below:
\begin{verbatim}
    \begin{figure}
    \centerline{\includegraphics[width=4.5in]{mouse.eps}}
    \caption{This is an example of figure caption.}
    \end{figure}
\end{verbatim}

%\figurebox{10pc}{10pc}{art.eps}
%[height][width][filename.eps]

Tables can be coded as below:
\begin{verbatim}
   \begin{table}[b]%[t]
   \def~{\hphantom{0}}%
   \tbl{This is an example of table caption.
   This is an example of table caption}{%
   \begin{tabular}{@{}cccc@{}}%
   \toprule
   First column & Second column & Third column & \\
   head$^{\rm a}$  &  head$^{\rm b}$  &  head 
   &  $V_M(r)$ \\
   \colrule
   Left    &  Word entries  &~0.2~ & 10.55 \\  
   Left    &  Word entries  &~0.15 &  33.12  \\ 
   Left    &  Word entries  &10.58 & 45.10 \\ 
   Left    &  Word entries  &43.9~ & 12.34 \\
   Left    &  Word entries  &~0.15 & 60.50 \\
   \botrule
   \end{tabular}}
   \tabnote{$^{\rm a}$This is a table footnote.
   This is a table footnote.}
   \tabnote{$^{\rm b}$This is a table footnote. 
   This is a table footnote.}
   \end{table}%
\end{verbatim}
\verb|\tbl{}{}| is not part of the standard \LaTeX\ package. 

\subsubsection{List environment}

For lists of items that are complete and/or multiple sentences, use numbers with only the first
line indented. An example of a list is as follows:
\begin{enumerate}
\item{} This is an example of the first item on a list, but note that only the first line of each item on the list is indented.
\item{} Item two follows item one.
\item{} Item three, and any other items follow until the end of the list.
\end{enumerate}
Alternatively, for lists of items given in sentence fragments, use numbers within the paragraph.
An example is as follows: 1) Item one, 2) Item two, 3) Item three. The second type of list is preferred, and
most lists will be converted to this form by the typesetter.

\begin{verbatim}
    \begin{enumerate}
      \item{} This is an example of the first item on a list.
      \item{} Item two follows item one.
      \item{} Item three, and any other items follow until the end of the list.
    \end{enumerate}
\end{verbatim}

\subsection{Equations, Numbers, Symbols, and Abbreviations}

Equations are centered and numbered consecutively, with equation
numbers in parentheses flush right, as in Eq. (1). Equation (1) is
coded as below:
\begin{verbatim}
   \begin{equation}
   \int_0^{r_2} \bm{F}(r,\varphi)\,dr\,d\varphi= 
   [\sigma r_2/(2\mu_0)]\cdot \int_0^{\infty}
   \exp(-\rho|z_j-z_i|)\lambda^{-1} 
   \end{equation}
\end{verbatim}
A sample equation is included below, formatted using the preceding
instructions. To make your equation more compact, you can use the
solidus ($/$), the exp function, or appropriate exponents.  Use
parentheses to avoid ambiguities in denominators.
\begin{equation}
\int_0^{r_2} \bm{F}(r,\varphi)\,dr\,d\varphi= [\sigma r_2/(2\mu_0)]
\cdot \int_0^{\infty}\exp(-\rho|z_j-z_i|)\lambda^{-1} 
\end{equation}
Be sure that the symbols in your equation are defined in the Nomenclature or before the
equation appears, or immediately following. Italicize symbols ($T$
might refer to temperature, but T is the unit Tesla). Refer to ``Eq.
(1),'' not ``(1)'' or ``equation (1)'' except at the beginning of a
sentence use: ``Equation (1) is.'' Equations can be labeled other than
``Eq.'' but should they represent inequalities, matrices, or boundary
conditions. If what is represented is really more than one equation,
the abbreviation ``Eqs.'' can be used.


\section{Conclusions}
The most important results of the paper should be summarized as a concise list of numbered items.
Conclusions should be supported by development in the main text and no new material should be
introduced in this section. If the paper did not result in specific conclusions, then the section may be entitled
Concluding Remarks or Recommendations, with brief summary comments as appropriate. No references or
equations must be cited in the conclusions section.

\section{Appendices}
An appendix or appendices should be included only for highly specialized data, derivations, and so forth.
Appendices should not be used to define symbols. The appendices should be numbered if more than one is
used. Each appendix must be cited in the main text.

\section{Acknowledgments}

The preferred spelling of the word ``acknowledgments'' in American
English is without the ``e'' after the ``g.'' Avoid expressions such as
``One of us (S.B.A.) would like to thank�'' Instead, write ``F. A.
Author thanks�'' Sponsor and financial support acknowledgments are
also to be listed in the ``acknowledgments'' section.
%\vfill\pagebreak


\widowpenalty=10000
\clubpenalty=10000

\section{References}
The following pages are intended to provide examples of the different
reference types, as used in the AHS Style Guide.  The bibliographic
(or reference) chapter is coded within the environment
\verb|{thebibliography}|.
\begin{verbatim}
   \begin{thebibliography}{9}%%% Maximum refs 
   \bibitem{Johnson80}
   Johnson, W., {\it Helicopter Theory}, Princeton University Press, 
   Princeton, NJ, 1980, pp. 808--813.

   \bibitem{Leishman00}
   Leishman, J. G., {\it Principles of Helicopter Aerodynamics}, 
   Cambridge University Press, New York, NY, 2000, Chapter 10.
   ...
   ...
   \end{thebibliography}
   \end{verbatim}

If you are using a printed or PDF version of this document,
all references should be in 12-point font, with reference numbers
inserted in superscript immediately before the corresponding
reference. You are not required to indicate the type of reference;
different types are shown here for illustrative purposes only.\medskip

\begin{thebibliography}{9}

\bibitem{Johnson80a}
Johnson, W., {\it Helicopter Theory}, Princeton University Press, 
Princeton, NJ, 1980, pp. 808--813.

\bibitem{Leishman00}
Leishman, J. G., {\it Principles of Helicopter Aerodynamics}, 
Cambridge University Press, New York, NY, 2000, Chapter 10.

\bibitem{Friedmann93}
Friedmann, P. P., and Hodges, D. H., ``Rotary-Wing Aeroelasticity with Application to VTOL Vehicles,''
{\it Flight-Vehicle Materials, Structures, and Dynamics}, edited by A. K. Noor and S. L. Venneri, Vol. 5, Part II,
Chap. 6, American Society of Mechanical Engineers, New York, NY, 1993, pp. 299--391.

\bibitem{Chopra85}
Chopra, I., ``Dynamic Stability of a Bearingless Circulation 
Control Rotor Blade in Hover,'' {\it Journal of the American 
Helicopter Society}, Vol. 30, (4), Oct. 1985, pp. 40--47.

\bibitem{Marchman72}
Marchman, J. F., III, and Uzel, J. N., ``Effect of Several 
Wing Tip Modifications on a Trailing Vortex,''
{\it Journal of Aircraft}, Vol. 9, (9), 1972, pp. 684--686.

\bibitem{Carpenter53}
Carpenter, P. J., and Friedovich, B., ``Effect of a Rapid Blade-Pitch Increase on the Thrust and
Induced-Velocity Response of a Full-Scale Helicopter Rotor,'' NACA TN 3044, 1953.

\bibitem{Johnson80b}
Johnson, W., ``A Comprehensive Analytical Model of Rotorcraft Aerodynamics and Dynamics,
Part I: Analytical Development,'' NASA TM 81182, 1980.

\bibitem{Sadler71}
Sadler, S. G., ``A Method for Predicting Helicopter Wake Geometry, Wake-Induced Inflow and
Wake Effects on Blade Airloads,'' American Helicopter Society 27th Annual Forum Proceedings,
Washington, DC, May 1971.

\bibitem{Brentner00}
Brentner, K. S., and Jones, H. E., ``Noise Prediction for Maneuvering Rotorcraft,'' Paper AIAA
2000--2031, 6th AIAA/CEAS Aeroacoustics Conference Proceedings, Lahaina, HI, June 12--14,
2000.
\end{thebibliography}

\listoffigures
\listoftables

\begin{figure}
\centerline{\includegraphics[]{mouse.eps}}
\caption{This is an example of a figure caption.}
\end{figure}

\clearpage

\begin{figure}
\centerline{\includegraphics[width=4.5in]{mouse.eps}}
\caption{This is an example of a figure caption.}
\end{figure}

\clearpage

\begin{table}
\def~{\hphantom{0}}%
\tbl{This is an example of a table caption.}{
\begin{tabular}{@{}cccc@{}}%
\toprule
First column & Second column & Third column & \\
head$^{\rm a}$  &  head$^{\rm b}$  &  head&  $V_M(r)$ \\
\colrule
Left    &  Word entries  & ~0.2 & 10.55 \\  
Left    &  Word entries  & 10.1 & 33.12  \\ 
Left    &  Word entries  & ~0.5 & 45.10 \\ 
Left    &  Word entries  & 13.9 & 12.34 \\
Left    &  Word entries  & ~0.1 & 60.50 \\
\botrule
\end{tabular}}
\tabnote{$^{\rm a}$This is a table footnote.}
\tabnote{$^{\rm b}$This is a table footnote.}
\end{table}

\clearpage

\begin{table}
\def~{\hphantom{0}}
\tbl{This is an example of a table caption.}{
\begin{tabular}{@{}cccccc@{}}
\toprule
First column   & Second column  & Third column & Forth column & Sixth column & \\
head$^{\rm a}$ & head$^{\rm b}$ & head         & head         & head         & $V_M(r)$ \\
\colrule
Left           &  Word entries  & ~0.2~        & 10.55        & 89--99       & 12\\  
Left           &  Word entries  & ~0.15        & 33.12        & 56--87       & 15\\ 
Left           &  Word entries  & 10.58        & 45.10        & 65--78       & 20\\ 
Left           &  Word entries  & 43.9~        & 12.34        & 89--92       & 19\\
Left           &  Word entries  & ~0.15        & 60.50        & 91--99       & 20\\
\botrule
\end{tabular}}
\tabnote{$^{\rm a}$This is a table footnote.}
\tabnote{$^{\rm b}$This is a table footnote.}
\end{table}%

\end{document}
