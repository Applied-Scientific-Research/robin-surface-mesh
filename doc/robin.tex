\documentclass[12pt]{article}
\usepackage[dvips]{graphicx}
\usepackage{amsmath}
\usepackage{array}
\usepackage{natbib}
\usepackage{latexsym}
%\usepackage{fancyvrb}
%\usepackage{dcolumn}
% more text vertically
\addtolength{\textheight}{0.5in}
\addtolength{\topmargin}{-0.5in}
% more text horizontally
\setlength{\oddsidemargin}{0.25in}
\setlength{\evensidemargin}{0.25in}
\addtolength{\textwidth}{0.5in}
% more closely-positioned figures
\renewcommand{\topfraction}{0.75}
\renewcommand{\bottomfraction}{0.5}
\renewcommand{\textfraction}{0.25}
\renewcommand{\floatpagefraction}{0.75}
 %
\title{Robust and Corrected Coefficients for the ROBIN Body}
 %
\author{Mark J.~Stock and Blake B.~Hillier\\
        Applied Scientific Research, Inc.
        Irvine, CA\\
        \texttt{markjstock@gmail.com}, \texttt{blakehillier@mac.com}
}
\begin{document}
\maketitle
%\begin{abstract}
%\end{abstract}

% --------------------------------------------------------------------------

\abstract{
The ROBIN body is a VTOL fuselage shape \cite{nasa87762,mineckgorton,nasa80051,nasa1999}
which is defined by the intersection of multiple sets of analytic equations.
While the shape is used frequently in rotorcraft aerodynamics research, the coefficients,
where published, do not reliably produce the desired geometry.
This letter serves as a notice of the correct coefficients and a public repository
for their distribution \cite{robinsurfmesh}.
}

% --------------------------------------------------------------------------
\section{Proposed changes}
The superellipse function used to define the ROBIN body, in its most common form, is as follows.
\begin{equation}
  F(x) = C_{6}+C_{7}\left[C_{1}+C_{2}\left(\frac{x+C_{3}}{C_{4}}\right)^{C_{5}}\right]^{\frac{1}{C_{8}}}
\end{equation}
In order to facilitate correct computation of the shape, certain changes to the
widely-published coefficients had to be made.
Figures 1 and 2 tabulate these corrections.
The changes fall into several categories, summarized below.
\begin{itemize}
\item Denominators $C_{4}$ in the front section of both the fuselage and pylon had the wrong sign
and created incorrect results.
\item The denominator in the outermost exponent ($C_{8}$) was changed from 0 to 1. Even though
those formulae do not need to evaluate the contents of the base (because $C_{7}$ is also zero),
these exponents should not throw NaN or Inf when evaluated. This was point out in \cite{nasa87762}.
\item Similarly, we changed $C_{4}$ from 0 to 1 to prevent division by zero, even though
$C_{2}=C_{5}=0$ meant that there was no need to evaluate the division.
\item Rows with constant values placed the value in $C_{1}$, but were multiplied by $C_{7}=0$,
these single values should more clearly be placed in the outermost constant $C_{6}$.
\item Finally, some obvious typos ($Z_0$ in the second body segment, $N$ on the pylon aft)
in the original \cite{nasa80051} were
caught by later authors \cite{nasa87762,mineckgorton}, but no author caught both.
\end{itemize}

A final note is in order: components guiding the shape of the pylon ($C_{7}$) have differed
between earlier and later works. We have chosen to use values from Mineck and Gorton (2000)
\cite{mineckgorton} because the shape of the pylon generated more closely matches the photos
and drawings from earlier papers \cite{nasa80051,nasa87762} than the original coefficients.

\begin{figure} \begin{centering}
\begin{small}
\begin{tabular}{cccccccccc}
Function & x & $C_{1}$ & $C_{2}$ & $C_{3}$ & $C_{4}$ & $C_{5}$ & $C_{6}$ & $C_{7}$ & $C_{8}$ \\
\hline
H          & [0.0, 0.4]  & 1.0 & -1.0 & -0.4 & \textbf{-0.4} & 1.8 & 0.0    & 0.25 & 1.8 \\
W          &                 & 1.0 & -1.0 & -0.4 & \textbf{-0.4} & 2.0 & 0.0    & 0.25 & 2.0 \\
$Z_{0}$ &                 & 1.0 & -1.0 & -0.4 & \textbf{-0.4} & 1.8 & -0.08 & 0.08 & 1.8 \\
N           &                 & 2.0 & 3.0  & 0.0  &              0.4  & 1.0 & 0.0    & 1.0   & 1.0 \\
\hline
H          & [0.4, 0.8]  & \textbf{0.0} & 0.0 & 0.0 & \textbf{1.0} & 0.0 & \textbf{0.25} & 0.0 & \textbf{1.0} \\
W          &                 & \textbf{0.0} & 0.0 & 0.0 & \textbf{1.0} & 0.0 & \textbf{0.25} & 0.0 & \textbf{1.0} \\
$Z_{0}$ &                 & \textbf{0.0} & 0.0 & 0.0 & \textbf{1.0} & 0.0 & 0.0               & 0.0 & \textbf{1.0} \\
N           &                 & \textbf{0.0} & 0.0 & 0.0 & \textbf{1.0} & 0.0 & \textbf{5.0}   & 0.0 & \textbf{1.0} \\
\hline
H          & [0.8, 1.9]  & 1.0 & -1.0 & -0.8 & 1.1 & 1.5 & 0.05 & 0.2             & 0.6 \\
W.         &                 & 1.0 & -1.0 & -0.8 & 1.1 & 1.5 & 0.05 & 0.2             & 0.6 \\
$Z_{0}$ &                 & 1.0 & -1.0 & -0.8 & 1.1 & 1.5 & 0.04 & -0.04        & 0.6 \\
N           &                 & 5.0 & -3.0 & -0.8 & 1.1 & 1.0 & 0.0   & \textbf{1.0} & \textbf{1.0} \\
\hline
H          & [1.9, 2.0]  & 1.0             & -1.0 & \textbf{-0.8} & \textbf{1.1} & \textbf{1.5} & \textbf{0.05} & \textbf{0.2} & \textbf{0.6} \\
W          &                 & 1.0             & -1.0 &             -1.9 &             0.1 &             2.0 &             0.0   & 0.05           & 2.0 \\
$Z_{0}$ &                 & \textbf{0.0} & 0.0  &             0.0  & \textbf{1.0} &             0.0 & \textbf{0.04} & 0.0             & \textbf{1.0} \\
N           &                 & \textbf{0.0} & 0.0  &             0.0  & \textbf{1.0} &             0.0 & \textbf{2.0}   & 0.0             & \textbf{1.0} \\
\end{tabular}
%\vspace{0.1in}
\caption{Coefficients for the fuselage, changes are in \textbf{bold}}
\label{fuscoeff}
\end{small}
\end{centering}\end{figure}%

\begin{figure} \begin{centering}
\begin{small}
\begin{tabular}{cccccccccc}
Function & x & $C_{1}$ & $C_{2}$ & $C_{3}$ & $C_{4}$ & $C_{5}$ & $C_{6}$ & $C_{7}$ & $C_{8}$ \\
\hline
H          & [0.4, 0.8]  & 1.0             & -1.0 & -0.8 & \textbf{-0.4} & 3.0 & 0.0                  & \textbf{0.145} & 3.0 \\
W          &                 & 1.0             & -1.0 & -0.8 & \textbf{-0.4} & 3.0 & 0.0                  & \textbf{0.166} & 3.0 \\
$Z_{0}$ &                 & \textbf{0.0} & 0.0  & 0.0  & \textbf{1.0}  & 0.0  & \textbf{0.125} & 0.0                 & \textbf{1.0} \\
N           &                 & \textbf{0.0} & 0.0  & 0.0  & \textbf{1.0}  & 0.0  & \textbf{5.0}     & 0.0                 & \textbf{1.0} \\
\hline
H          & [0.8, 1.018]  & 1.0             & -1.0 & -0.8 & 0.218         & 2.0 & 0.0                 & \textbf{0.145} & 2.0 \\
W          &                     & 1.0             & -1.0 & -0.8 & 0.218         & 2.0 & 0.0                 & \textbf{0.166} & 2.0 \\
$Z_{0}$ &                     & 1.0             & -1.0 & -0.8 & 1.1             & 1.5 & 0.065             & 0.06               & 0.6 \\
N           &                     & \textbf{0.0} & 0.0  & 0.0  & \textbf{1.0} & 0.0 & \textbf{5.0}     & 0.0                 & \textbf{1.0} \\
\end{tabular}
%\vspace{0.1in}
\caption{Coefficients for the pylons, changes are in \textbf{bold}}
\label{pycoeff}
\end{small}
\end{centering}\end{figure}%

% --------------------------------------------------------------------------
\bibliographystyle{abbrv}
\bibliography{robinbib}
\end{document}
%%
